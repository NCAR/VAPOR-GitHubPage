% Generated by Sphinx.
\def\sphinxdocclass{report}
\documentclass[letterpaper,10pt,english]{sphinxmanual}
\usepackage{iftex}

\ifPDFTeX
  \usepackage[utf8]{inputenc}
\fi
\ifdefined\DeclareUnicodeCharacter
  \DeclareUnicodeCharacter{00A0}{\nobreakspace}
\fi
\usepackage{cmap}
\usepackage[T1]{fontenc}
\usepackage{amsmath,amssymb,amstext}
\usepackage{babel}
\usepackage{times}
\usepackage[Bjarne]{fncychap}
\usepackage{longtable}
\usepackage{sphinx}
\usepackage{multirow}
\usepackage{eqparbox}


\addto\captionsenglish{\renewcommand{\figurename}{Fig.\@ }}
\addto\captionsenglish{\renewcommand{\tablename}{Table }}
\SetupFloatingEnvironment{literal-block}{name=Listing }

\addto\extrasenglish{\def\pageautorefname{page}}

\setcounter{tocdepth}{1}


\title{DART Documentation}
\date{Feb 06, 2019}
\release{2.0.0}
\author{DART Team}
\newcommand{\sphinxlogo}{}
\renewcommand{\releasename}{Release}
\makeindex

\makeatletter
\def\PYG@reset{\let\PYG@it=\relax \let\PYG@bf=\relax%
    \let\PYG@ul=\relax \let\PYG@tc=\relax%
    \let\PYG@bc=\relax \let\PYG@ff=\relax}
\def\PYG@tok#1{\csname PYG@tok@#1\endcsname}
\def\PYG@toks#1+{\ifx\relax#1\empty\else%
    \PYG@tok{#1}\expandafter\PYG@toks\fi}
\def\PYG@do#1{\PYG@bc{\PYG@tc{\PYG@ul{%
    \PYG@it{\PYG@bf{\PYG@ff{#1}}}}}}}
\def\PYG#1#2{\PYG@reset\PYG@toks#1+\relax+\PYG@do{#2}}

\expandafter\def\csname PYG@tok@gd\endcsname{\def\PYG@tc##1{\textcolor[rgb]{0.63,0.00,0.00}{##1}}}
\expandafter\def\csname PYG@tok@gu\endcsname{\let\PYG@bf=\textbf\def\PYG@tc##1{\textcolor[rgb]{0.50,0.00,0.50}{##1}}}
\expandafter\def\csname PYG@tok@gt\endcsname{\def\PYG@tc##1{\textcolor[rgb]{0.00,0.27,0.87}{##1}}}
\expandafter\def\csname PYG@tok@gs\endcsname{\let\PYG@bf=\textbf}
\expandafter\def\csname PYG@tok@gr\endcsname{\def\PYG@tc##1{\textcolor[rgb]{1.00,0.00,0.00}{##1}}}
\expandafter\def\csname PYG@tok@cm\endcsname{\let\PYG@it=\textit\def\PYG@tc##1{\textcolor[rgb]{0.25,0.50,0.56}{##1}}}
\expandafter\def\csname PYG@tok@vg\endcsname{\def\PYG@tc##1{\textcolor[rgb]{0.73,0.38,0.84}{##1}}}
\expandafter\def\csname PYG@tok@vi\endcsname{\def\PYG@tc##1{\textcolor[rgb]{0.73,0.38,0.84}{##1}}}
\expandafter\def\csname PYG@tok@vm\endcsname{\def\PYG@tc##1{\textcolor[rgb]{0.73,0.38,0.84}{##1}}}
\expandafter\def\csname PYG@tok@mh\endcsname{\def\PYG@tc##1{\textcolor[rgb]{0.13,0.50,0.31}{##1}}}
\expandafter\def\csname PYG@tok@cs\endcsname{\def\PYG@tc##1{\textcolor[rgb]{0.25,0.50,0.56}{##1}}\def\PYG@bc##1{\setlength{\fboxsep}{0pt}\colorbox[rgb]{1.00,0.94,0.94}{\strut ##1}}}
\expandafter\def\csname PYG@tok@ge\endcsname{\let\PYG@it=\textit}
\expandafter\def\csname PYG@tok@vc\endcsname{\def\PYG@tc##1{\textcolor[rgb]{0.73,0.38,0.84}{##1}}}
\expandafter\def\csname PYG@tok@il\endcsname{\def\PYG@tc##1{\textcolor[rgb]{0.13,0.50,0.31}{##1}}}
\expandafter\def\csname PYG@tok@go\endcsname{\def\PYG@tc##1{\textcolor[rgb]{0.20,0.20,0.20}{##1}}}
\expandafter\def\csname PYG@tok@cp\endcsname{\def\PYG@tc##1{\textcolor[rgb]{0.00,0.44,0.13}{##1}}}
\expandafter\def\csname PYG@tok@gi\endcsname{\def\PYG@tc##1{\textcolor[rgb]{0.00,0.63,0.00}{##1}}}
\expandafter\def\csname PYG@tok@gh\endcsname{\let\PYG@bf=\textbf\def\PYG@tc##1{\textcolor[rgb]{0.00,0.00,0.50}{##1}}}
\expandafter\def\csname PYG@tok@ni\endcsname{\let\PYG@bf=\textbf\def\PYG@tc##1{\textcolor[rgb]{0.84,0.33,0.22}{##1}}}
\expandafter\def\csname PYG@tok@nl\endcsname{\let\PYG@bf=\textbf\def\PYG@tc##1{\textcolor[rgb]{0.00,0.13,0.44}{##1}}}
\expandafter\def\csname PYG@tok@nn\endcsname{\let\PYG@bf=\textbf\def\PYG@tc##1{\textcolor[rgb]{0.05,0.52,0.71}{##1}}}
\expandafter\def\csname PYG@tok@no\endcsname{\def\PYG@tc##1{\textcolor[rgb]{0.38,0.68,0.84}{##1}}}
\expandafter\def\csname PYG@tok@na\endcsname{\def\PYG@tc##1{\textcolor[rgb]{0.25,0.44,0.63}{##1}}}
\expandafter\def\csname PYG@tok@nb\endcsname{\def\PYG@tc##1{\textcolor[rgb]{0.00,0.44,0.13}{##1}}}
\expandafter\def\csname PYG@tok@nc\endcsname{\let\PYG@bf=\textbf\def\PYG@tc##1{\textcolor[rgb]{0.05,0.52,0.71}{##1}}}
\expandafter\def\csname PYG@tok@nd\endcsname{\let\PYG@bf=\textbf\def\PYG@tc##1{\textcolor[rgb]{0.33,0.33,0.33}{##1}}}
\expandafter\def\csname PYG@tok@ne\endcsname{\def\PYG@tc##1{\textcolor[rgb]{0.00,0.44,0.13}{##1}}}
\expandafter\def\csname PYG@tok@nf\endcsname{\def\PYG@tc##1{\textcolor[rgb]{0.02,0.16,0.49}{##1}}}
\expandafter\def\csname PYG@tok@si\endcsname{\let\PYG@it=\textit\def\PYG@tc##1{\textcolor[rgb]{0.44,0.63,0.82}{##1}}}
\expandafter\def\csname PYG@tok@s2\endcsname{\def\PYG@tc##1{\textcolor[rgb]{0.25,0.44,0.63}{##1}}}
\expandafter\def\csname PYG@tok@nt\endcsname{\let\PYG@bf=\textbf\def\PYG@tc##1{\textcolor[rgb]{0.02,0.16,0.45}{##1}}}
\expandafter\def\csname PYG@tok@nv\endcsname{\def\PYG@tc##1{\textcolor[rgb]{0.73,0.38,0.84}{##1}}}
\expandafter\def\csname PYG@tok@s1\endcsname{\def\PYG@tc##1{\textcolor[rgb]{0.25,0.44,0.63}{##1}}}
\expandafter\def\csname PYG@tok@dl\endcsname{\def\PYG@tc##1{\textcolor[rgb]{0.25,0.44,0.63}{##1}}}
\expandafter\def\csname PYG@tok@ch\endcsname{\let\PYG@it=\textit\def\PYG@tc##1{\textcolor[rgb]{0.25,0.50,0.56}{##1}}}
\expandafter\def\csname PYG@tok@m\endcsname{\def\PYG@tc##1{\textcolor[rgb]{0.13,0.50,0.31}{##1}}}
\expandafter\def\csname PYG@tok@gp\endcsname{\let\PYG@bf=\textbf\def\PYG@tc##1{\textcolor[rgb]{0.78,0.36,0.04}{##1}}}
\expandafter\def\csname PYG@tok@sh\endcsname{\def\PYG@tc##1{\textcolor[rgb]{0.25,0.44,0.63}{##1}}}
\expandafter\def\csname PYG@tok@ow\endcsname{\let\PYG@bf=\textbf\def\PYG@tc##1{\textcolor[rgb]{0.00,0.44,0.13}{##1}}}
\expandafter\def\csname PYG@tok@sx\endcsname{\def\PYG@tc##1{\textcolor[rgb]{0.78,0.36,0.04}{##1}}}
\expandafter\def\csname PYG@tok@bp\endcsname{\def\PYG@tc##1{\textcolor[rgb]{0.00,0.44,0.13}{##1}}}
\expandafter\def\csname PYG@tok@c1\endcsname{\let\PYG@it=\textit\def\PYG@tc##1{\textcolor[rgb]{0.25,0.50,0.56}{##1}}}
\expandafter\def\csname PYG@tok@fm\endcsname{\def\PYG@tc##1{\textcolor[rgb]{0.02,0.16,0.49}{##1}}}
\expandafter\def\csname PYG@tok@o\endcsname{\def\PYG@tc##1{\textcolor[rgb]{0.40,0.40,0.40}{##1}}}
\expandafter\def\csname PYG@tok@kc\endcsname{\let\PYG@bf=\textbf\def\PYG@tc##1{\textcolor[rgb]{0.00,0.44,0.13}{##1}}}
\expandafter\def\csname PYG@tok@c\endcsname{\let\PYG@it=\textit\def\PYG@tc##1{\textcolor[rgb]{0.25,0.50,0.56}{##1}}}
\expandafter\def\csname PYG@tok@mf\endcsname{\def\PYG@tc##1{\textcolor[rgb]{0.13,0.50,0.31}{##1}}}
\expandafter\def\csname PYG@tok@err\endcsname{\def\PYG@bc##1{\setlength{\fboxsep}{0pt}\fcolorbox[rgb]{1.00,0.00,0.00}{1,1,1}{\strut ##1}}}
\expandafter\def\csname PYG@tok@mb\endcsname{\def\PYG@tc##1{\textcolor[rgb]{0.13,0.50,0.31}{##1}}}
\expandafter\def\csname PYG@tok@ss\endcsname{\def\PYG@tc##1{\textcolor[rgb]{0.32,0.47,0.09}{##1}}}
\expandafter\def\csname PYG@tok@sr\endcsname{\def\PYG@tc##1{\textcolor[rgb]{0.14,0.33,0.53}{##1}}}
\expandafter\def\csname PYG@tok@mo\endcsname{\def\PYG@tc##1{\textcolor[rgb]{0.13,0.50,0.31}{##1}}}
\expandafter\def\csname PYG@tok@kd\endcsname{\let\PYG@bf=\textbf\def\PYG@tc##1{\textcolor[rgb]{0.00,0.44,0.13}{##1}}}
\expandafter\def\csname PYG@tok@mi\endcsname{\def\PYG@tc##1{\textcolor[rgb]{0.13,0.50,0.31}{##1}}}
\expandafter\def\csname PYG@tok@kn\endcsname{\let\PYG@bf=\textbf\def\PYG@tc##1{\textcolor[rgb]{0.00,0.44,0.13}{##1}}}
\expandafter\def\csname PYG@tok@cpf\endcsname{\let\PYG@it=\textit\def\PYG@tc##1{\textcolor[rgb]{0.25,0.50,0.56}{##1}}}
\expandafter\def\csname PYG@tok@kr\endcsname{\let\PYG@bf=\textbf\def\PYG@tc##1{\textcolor[rgb]{0.00,0.44,0.13}{##1}}}
\expandafter\def\csname PYG@tok@s\endcsname{\def\PYG@tc##1{\textcolor[rgb]{0.25,0.44,0.63}{##1}}}
\expandafter\def\csname PYG@tok@kp\endcsname{\def\PYG@tc##1{\textcolor[rgb]{0.00,0.44,0.13}{##1}}}
\expandafter\def\csname PYG@tok@w\endcsname{\def\PYG@tc##1{\textcolor[rgb]{0.73,0.73,0.73}{##1}}}
\expandafter\def\csname PYG@tok@kt\endcsname{\def\PYG@tc##1{\textcolor[rgb]{0.56,0.13,0.00}{##1}}}
\expandafter\def\csname PYG@tok@sc\endcsname{\def\PYG@tc##1{\textcolor[rgb]{0.25,0.44,0.63}{##1}}}
\expandafter\def\csname PYG@tok@sb\endcsname{\def\PYG@tc##1{\textcolor[rgb]{0.25,0.44,0.63}{##1}}}
\expandafter\def\csname PYG@tok@sa\endcsname{\def\PYG@tc##1{\textcolor[rgb]{0.25,0.44,0.63}{##1}}}
\expandafter\def\csname PYG@tok@k\endcsname{\let\PYG@bf=\textbf\def\PYG@tc##1{\textcolor[rgb]{0.00,0.44,0.13}{##1}}}
\expandafter\def\csname PYG@tok@se\endcsname{\let\PYG@bf=\textbf\def\PYG@tc##1{\textcolor[rgb]{0.25,0.44,0.63}{##1}}}
\expandafter\def\csname PYG@tok@sd\endcsname{\let\PYG@it=\textit\def\PYG@tc##1{\textcolor[rgb]{0.25,0.44,0.63}{##1}}}

\def\PYGZbs{\char`\\}
\def\PYGZus{\char`\_}
\def\PYGZob{\char`\{}
\def\PYGZcb{\char`\}}
\def\PYGZca{\char`\^}
\def\PYGZam{\char`\&}
\def\PYGZlt{\char`\<}
\def\PYGZgt{\char`\>}
\def\PYGZsh{\char`\#}
\def\PYGZpc{\char`\%}
\def\PYGZdl{\char`\$}
\def\PYGZhy{\char`\-}
\def\PYGZsq{\char`\'}
\def\PYGZdq{\char`\"}
\def\PYGZti{\char`\~}
% for compatibility with earlier versions
\def\PYGZat{@}
\def\PYGZlb{[}
\def\PYGZrb{]}
\makeatother

\renewcommand\PYGZsq{\textquotesingle}

\begin{document}

\maketitle
\tableofcontents
\phantomsection\label{index::doc}


Contents:


\chapter{DART Lanai Differences from Kodiak Release Notes}
\label{docs/Lanai_diffs_from_Kodiak::doc}\label{docs/Lanai_diffs_from_Kodiak:dart-lanai-differences-from-kodiak-release-notes}\label{docs/Lanai_diffs_from_Kodiak:welcome-to-dart-s-documentation}

\section{Overview}
\label{docs/Lanai_diffs_from_Kodiak:overview}
This document includes an overview of the changes in the DART system
since the Kodiak release. For further details on any of these items look
at the HTML documentation for that specific part of the system.

There is a longer companion document for this release, the Lanai
Release Notes, which include installation instructions, a walk-through of running one
of the low-order models, the diagnostics, and a description of
non-backward compatible changes. See the Notes for Current
Users section for additional information on changes in this release.


\bigskip\hrule{}\bigskip



\section{Changes to Core DART routines}
\label{docs/Lanai_diffs_from_Kodiak:changes-to-core-dart-routines}
This section describes changes in the basic DART library routines since
the Kodiak release.
\begin{itemize}
\item {} 
Added a completely new random number generator based on the Mersenne
Twister algorithm from the GNU scientific library. It seems to have
better behavior if reseeded frequently, which is a possible usage
pattern if perfect\_model\_obs is run for only single steps and the
model is advanced in an external script. As part of this code update
all random number code was moved into the random\_seq\_mod and
random\_nr\_mod is deprecated.

\item {} 
Perfect\_model\_obs calls a seed routine in the time manager now
that generates a consistent seed based on the current time of the
state. This makes subsequent runs give consistent results and yet
separate runs don`t get identical error values.

\item {} 
Added random number generator seeds in several routines to try to
get consistent results no matter how many MPI tasks the code was run
with. This includes:
\begin{itemize}
\item {} 
cam model\_mod.f90, pert\_model\_state()

\item {} 
assim\_tools\_mod.f90, filter\_assim(), filter kinds 2, 3, and 5

\item {} 
wrf model\_mod.f90, pert\_model\_state()

\item {} 
adaptive\_inflate\_mod.f90, adaptive\_inflate\_init(),
non-deterministic inf

\end{itemize}

\item {} 
There is a new \&filter\_nml namelist item:
enable\_special\_outlier\_code. If .true. the DART quality control
code will call a separate subroutine at the end of filter.f90 to
evaluate the outlier threshold. The user can add code to that
routine to change the threshold based on observation type or values
as they wish. If .false. the default filter outlier threshold code
will be called and the user routine ignored.

\item {} 
If your \emph{model\_mod.f90} provides a customized \emph{get\_close\_obs()}
routine that makes use of the types/kinds arguments for either the
base location or the close location list, there is an important
change in this release. The fifth argument to the
\emph{get\_close\_obs()} call is now a list of generic kinds
corresponding to the location list. The fourth argument to the
\emph{get\_dist()} routine is now also a generic kind and not a specific
type. In previous versions of the system the list of close locations
was sometimes a list of specific types and other times a list of
generic kinds. The system now always passes generic kinds for the
close locations list for consistency. The base location and specific
type remains the same as before. If you have a \emph{get\_close\_obs()}
routine in your \emph{model\_mod.f90} file and have questions about
usage, \href{mailto:dart@ucar.edu}{contact} the DART development team.

\item {} 
Filter will call the end\_model() subroutine in the model\_mod for
the first time. It should have been called all along, but was not.

\item {} 
Added a time sort routine in the time\_manager\_mod.

\item {} 
Avoid a pair of all-to-all transposes when setting the inflation
mean and sd from the namelist. The new code finds the task which has
the two copies and sets them directly without a transpose. The log
messages were also moved to the end of the routine - if you read in
the mean/sd values from a restart file the log messages that printed
out the min/max values needed to be after the read from the file.

\item {} 
Reordered the send/receive loops in the all-to-all transposes to
scale better on yellowstone.

\item {} 
Remove a state-vector size array from the stack in
read\_ensemble\_restart(). The array is now allocated only if needed
and then deallocated. The ensemble write routine was changed before
the Kodiak release but the same code in read was apparently not
changed simply as an oversight.

\item {} 
If the ensemble mean is selected to be written out in dart restart
file format, the date might not have been updated correctly. The
code was fixed to ensure the ensemble mean date in the file was
correct.

\item {} 
filter writes the ensemble size into the log file.

\item {} 
Reorganized the code in the section of obs\_model\_mod that prints
out the time windows, with and without verbose details. Should be
clearer if the next observation is in or out of the current
assimilation window, and if the model needs to advance or not.

\item {} 
Added a fill\_inflation\_restart utility which can write a file with
a fixed mean and sd, so the first step of a long assimilation run
can use the same `start\_from\_restart\_file` as subsequent steps.

\item {} 
Added new location module options:
\begin{itemize}
\item {} 
Channel coordinate system

\item {} 
{[}0-1{]} periodic 3D coordinate system

\item {} 
X,Y,Z 3D Cartesian coordinate system

\item {} 
2D annulus coordinate system

\end{itemize}

\end{itemize}


\subsection{\textbar{}}
\label{docs/Lanai_diffs_from_Kodiak:id1}

\bigskip\hrule{}\bigskip



\section{New Models or Changes to Existing Models}
\label{docs/Lanai_diffs_from_Kodiak:new-models-or-changes-to-existing-models}
Several new models have been incorporated into DART. This section
details both changes to existing models and descriptions of new models
that have been added since the Kodiak release.
\begin{itemize}
\item {} 
Support for components under the CESM framework:
\begin{itemize}
\item {} 
Added support for the Community Land Model (CLM).

\item {} 
Added support to run the Community Atmospheric Model (CAM) under
the CESM framework.

\item {} 
Added support for the CESM 1.1.1 release for CAM, POP, CLM:
includes experiment setup scripts and assimilation scripts.

\item {} 
CAM, POP, and/or CLM can be assimilated either individually or
in combination while running under the CESM framework. If
assimilating into multiple components, they are assimilated
sequentially with observations only affecting a single component
directly. Other components are indirectly affected through
interactions with the coupler.

\item {} 
Setup scripts are provided to configure a CESM experiment using
the multi-instance feature of CESM to support ensembles for
assimilation.

\item {} 
POP state vector contains potential temperature; observations
from the World Ocean Database are in-situ or sensible
temperature. The model\_mod now corrects for this.
\begin{itemize}
\item {} 
The state vector has all along contained potential
temperature and not in-situ (sensible) temperature. The
observations from the World Ocean Database are of sensible
temperature. Changed the specific kind in the model\_mod to
be \code{QTY\_POTENTIAL\_TEMPERATURE} and added new code to convert
from potential to in-situ temperature. Differences for even
the deeper obs (4-5km) is still small ( \textasciitilde{} 0.2 degree).
(in-situ or sensible temperature is what you measure with a
regular thermometer.)

\end{itemize}

\item {} 
Support for the SE core (HOMME) of CAM has been developed but
\textbf{is not} part of the current release. Contact the DART group
if you have an interest in running this configuration of CAM.

\end{itemize}

\item {} 
Changes to the WRF model\_mod:
\begin{itemize}
\item {} 
Allow advanced microphysics schemes (needed interpolation for 7
new kinds)

\item {} 
Interpolation in the vertical is done in log(p) instead of
linear pressure space. log(p) is the default, but a compile-time
variable can restore the linear interpolation.

\item {} 
Added support in the namelist to avoid writing updated fields
back into the wrf netcdf files. The fields are still updated
during the assimilation but the updated data is not written back
to the wrfinput file during the dart\_to\_wrf step.

\item {} 
Fixed an obscure bug in the vertical convert routine of the wrf
model\_mod that would occasionally fail to convert an obs. This
would make tiny differences in the output as the number of mpi
tasks change. No quantitative differences in the results but
they were not bitwise compatible before and they are again now.

\end{itemize}

\item {} 
Added support for the MPAS\_ATM and MPAS\_OCN models.
\begin{itemize}
\item {} 
Added interpolation routines for the voroni-tesselation grid
(roughly hexagonal)

\item {} 
Includes vertical conversion routines for vertical localization.

\item {} 
Added code to the mpas\_atm model to interpolate specific
humidity and pressure, so we can assimilate GPS obs now.

\end{itemize}

\item {} 
Added support for the `SQG` uniform PV two-surface QC+1 spectral
model.

\item {} 
Added support for a flux-transport solar dynamo model.

\item {} 
Added support for the GITM upper atmosphere model.

\item {} 
Added support for the NOAH land model.

\item {} 
Added support for the NAAPS model.

\item {} 
Added model\_mod interface code for the NOGAPS model to the SVN
repository.

\item {} 
Simple advection model:
\begin{itemize}
\item {} 
Fix where the random number seed is set in the
models/simple\_advection model\_mod - it needed to be sooner
than it was being called.

\end{itemize}

\end{itemize}


\bigskip\hrule{}\bigskip



\section{New or changed Forward Operators}
\label{docs/Lanai_diffs_from_Kodiak:new-or-changed-forward-operators}
This section describes changes to the Foward Operators and new Generic
Kinds or Specific Types that have been added since the Kodiak release.
\begin{itemize}
\item {} 
Many new kinds added to the DEFAULT\_obs\_kind\_mod.f90:
\begin{itemize}
\item {} 
QTY\_CANOPY\_WATER

\item {} 
QTY\_CARBON

\item {} 
QTY\_CLW\_PATH

\item {} 
QTY\_DIFFERENTIAL\_REFLECTIVITY

\item {} 
QTY\_DUST

\item {} 
QTY\_EDGE\_NORMAL\_SPEED

\item {} 
QTY\_FLASH\_RATE\_2D

\item {} 
QTY\_GRAUPEL\_VOLUME

\item {} 
QTY\_GROUND\_HEAT\_FLUX QTY\_HAIL\_MIXING\_RATIO

\item {} 
QTY\_HAIL\_NUMBER\_CONCENTR

\item {} 
QTY\_HAIL\_VOLUME QTY\_ICE QTY\_INTEGRATED\_AOD

\item {} 
QTY\_INTEGRATED\_DUST

\item {} 
QTY\_INTEGRATED\_SEASALT QTY\_INTEGRATED\_SMOKE

\item {} 
QTY\_INTEGRATED\_SULFATE

\item {} 
QTY\_LATENT\_HEAT\_FLUX

\item {} 
QTY\_LEAF\_AREA\_INDEX

\item {} 
QTY\_LEAF\_CARBON

\item {} 
QTY\_LEAF\_NITROGEN QTY\_LIQUID\_WATER

\item {} 
QTY\_MICROWAVE\_BRIGHT\_TEMP

\item {} 
QTY\_NET\_CARBON\_FLUX

\item {} 
QTY\_NET\_CARBON\_PRODUCTION

\item {} 
QTY\_NEUTRON\_INTENSITY

\item {} 
QTY\_NITROGEN QTY\_RADIATION

\item {} 
QTY\_ROOT\_CARBON

\item {} 
QTY\_ROOT\_NITROGEN

\item {} 
QTY\_SEASALT

\item {} 
QTY\_SENSIBLE\_HEAT\_FLUX

\item {} 
QTY\_SMOKE

\item {} 
QTY\_SNOWCOVER\_FRAC

\item {} 
QTY\_SNOW\_THICKNESS

\item {} 
QTY\_SNOW\_WATER

\item {} 
QTY\_SO2

\item {} 
QTY\_SOIL\_CARBON

\item {} 
QTY\_SOIL\_NITROGEN

\item {} 
QTY\_SPECIFIC\_DIFFERENTIAL\_PHASE

\item {} 
QTY\_STEM\_CARBON

\item {} 
QTY\_STEM\_NITROGEN

\item {} 
QTY\_SULFATE

\item {} 
QTY\_VORTEX\_WMAX

\item {} 
QTY\_WATER\_TABLE\_DEPTH

\item {} 
QTY\_WIND\_TURBINE\_POWER

\item {} 
plus slots 151-250 reserved for Chemistry (specifically
WRF-Chem) kinds

\end{itemize}

\item {} 
Added a forward operator for total precipitable water. It loops over
model levels so it can be used as an example of how to handle this
without having to hardcode the number of levels into the operator.

\item {} 
Added a forward operator (and obs\_seq file converter) for COSMOS
ground moisture observations.

\item {} 
Added a forward operator (and obs\_seq file converter) for MIDAS
observations of Total Electron Count.

\item {} 
Added a `set\_1d\_integral()` routine to the
obs\_def\_1d\_state\_mod.f90 forward operator for the low order
models. This subroutine isn`t used by filter but it would be needed
if someone wanted to write a standalone program to generate obs of
this type. We use this file as an example of how to write an obs
type that has metadata, but we need to give an example of how to set
the metadata if you aren`t using create\_obs\_sequence interactively
(e.g. your data is in netcdf and you have a separate converter
program.)

\end{itemize}


\bigskip\hrule{}\bigskip



\section{Observation Converters}
\label{docs/Lanai_diffs_from_Kodiak:observation-converters}
This section describes support for new observation types or sources that
have been added since the Kodiak release.
\begin{itemize}
\item {} 
Added an obs\_sequence converter for wind profiler data from MADIS.

\item {} 
Added an obs\_sequence converter for Ameriflux land
observations(latent heat flux, sensible heat flux, net ecosystem
production).

\item {} 
Added an obs\_sequence converter for MODIS snow coverage
measurements.

\item {} 
Added an obs\_sequence converter for COSMOS ground moisture
observations.

\item {} 
Added an obs\_sequence converter for MIDAS observations of Total
Electron Count.

\item {} 
Updated scripts for the GPS converter; added options to convert data
from multiple satellites.

\item {} 
More scripting support in the MADIS obs converters; more error
checks added to the rawin converter.

\item {} 
Added processing for wind profiler observation to the
wrf\_dart\_obs\_preprocess program.

\item {} 
Fix BUG in airs converter - the humidity obs are accumulated across
the layers and so the best location for them is the layer midpoint
and not on the edges (levels) as the temperature obs are. Also fixed
off-by-one error where the converter would make one more obs above
the requested top level.

\item {} 
Made gts\_to\_dart converter create separate obs types for surface
dewpoint vs obs aloft because they have different vertical
coordinates.

\item {} 
Converted mss commands to hpss commands for a couple observation
converter shell scripts (inc AIRS).

\item {} 
New matlab code to generate evenly spaced observations on the
surface of a sphere (e.g. the globe).

\item {} 
Added obs\_loop.f90 example file in obs\_sequence directory; example
template for how to construct special purpose obs\_sequence tools.

\item {} 
Change the default in the script for the prepbufr converter so it
will swap bytes, since all machines except ibms will need this now.

\item {} 
The `wrf\_dart\_obs\_preprocess` program now refuses to superob
observations that include the pole, since the simple averaging of
latitude and longitude that works everyplace else won`t work there.
Also treats observations near the prime meridian more correctly.

\end{itemize}


\bigskip\hrule{}\bigskip



\section{New or updated DART Diagnostics}
\label{docs/Lanai_diffs_from_Kodiak:new-or-updated-dart-diagnostics}
This section describes new or updated diagnostic routines that have been
added since the Kodiak release.
\begin{itemize}
\item {} 
Handle empty epochs in the obs\_seq\_to\_netcdf converter.

\item {} 
Added a matlab utility to show the output of a `hop` test (running a
model for a continuous period vs. stopping and restarting a run).

\item {} 
Improved the routine that computes axes tick values in plots with
multiple values plotted on the same plot.

\item {} 
The obs\_common\_subset program can select common observations from
up to 4 observation sequence files at a time.

\item {} 
Add code in obs\_seq\_verify to ensure that the ensemble members are
in the same order in all netcdf files.

\item {} 
Added support for the unstructured grids of mpas to our matlab
diagnostics.

\item {} 
Fix to writing of ReportTime in obs\_seq\_coverage.

\item {} 
Fixed logic in obs\_seq\_verify when determining the forecast lat.

\item {} 
Fixed loops inside obs\_seq\_coverage which were using the wrong
limits on the loops. Fixed writing of `ntimes` in output netcdf
variable.

\item {} 
The obs\_common\_subset tool supports comparing more than 2
obs\_seq.final files at a time, and will loop over sets of files.

\item {} 
Rewrote the algorithm in the obs\_selection tool so it had better
scaling with large numbers of obs.

\item {} 
Several improvements to the `obs\_diag` program:
\begin{itemize}
\item {} 
Added preliminary support for a list of `trusted obs` in the
obs\_diag program.

\item {} 
Can disable the rank histogram generation with a namelist item.

\item {} 
Can define height\_edges or heights in the namelist, but not
both.

\item {} 
The `rat\_cri` namelist item (critical ratio) has been
deprecated.

\end{itemize}

\item {} 
Extend obs\_seq\_verify so it can be used for forecasts from a
single member. minor changes to obs\_selection, obs\_seq\_coverage
and obs\_seq\_verify to support a single member.

\item {} 
Added Matlab script to read/print timestamps from binary dart
restart/ic files.

\item {} 
Default for obs\_seq\_to\_netcdf in all the namelists is now `one
big time bin` so you don`t have to know the exact timespan of an
obs\_seq.final file before converting to netCDF.

\end{itemize}


\bigskip\hrule{}\bigskip



\section{Tutorial, Scripting, Setup, Builds}
\label{docs/Lanai_diffs_from_Kodiak:tutorial-scripting-setup-builds}
This section describes updates and changes to the tutorial materials,
scripting, setup, and build information since the Kodiak release.
\begin{itemize}
\item {} 
The mkmf-generated Makefiles now take care of calling `fixsystem` if
needed so the mpi utilities code compiles without further user
intervention.

\item {} 
Make the default input.nml for the Lorenz 96 and Lorenz 63 model
gives good assimilation results. Rename the original input.nml to
input.workshop.nml. The workshop\_setup script renames it back
before doing anything else so this won`t break the workshop
instructions. Simplify all the workshop\_setup.csh scripts to do the
minimal work needed by the DART tutorial.

\item {} 
Updates to the models/template directory with the start of a full 3d
geophysical model template. Still under construction.

\item {} 
Move the pdf files in the tutorial directory up a level. Removed
framemaker source files because we no longer have access to a
working version of the Framemaker software. Moved routines that
generate figures and diagrams to a non-distributed directory of the
subversion repository.

\item {} 
Enable netCDF large file support in the work/input.nml for models
which are likely to have large state vectors.

\item {} 
Minor updates to the doc.css file, make pages look identical in the
safari and firefox browsers.

\item {} 
Added a utility that sorts and reformats namelists, culls all
comments to the bottom of the file. Useful for doing diffs and
finding duplicated namelists in a file.

\item {} 
Cleaned up mkmf files - removed files for obsolete platforms and
compilers, updated suggested default flags for intel.

\item {} 
Update the mkmf template for gfortran to allow fortran source lines
longer than 132 characters.

\end{itemize}


\bigskip\hrule{}\bigskip



\section{Terms of Use}
\label{docs/Lanai_diffs_from_Kodiak:terms-of-use}
DART software - Copyright UCAR. This open source software is provided by
UCAR, ``as is``, without charge, subject to all terms of use at
\url{http://www.image.ucar.edu/DAReS/DART/DART\_download}


\bigskip\hrule{}\bigskip


Contact:             DART core group
Revision:            \$Revision\$
Source:              \$URL\$
Change Date:         \$Date\$
Change history:      try ``svn log`` or ``svn diff``


\bigskip\hrule{}\bigskip



\chapter{Fortran}
\label{docs/auto-doc:fortran}\label{docs/auto-doc::doc}

\section{Closest Member Tool}
\label{docs/auto-doc:closest-member-tool}\index{closest\_member\_tool (fortran program)|textbf}

\begin{fulllineitems}
\phantomsection\label{docs/auto-doc:f/closest_member_tool}\pysigline{\strong{program  }\bfcode{closest\_member\_tool}}
Program to overwrite the time on each ensemble in a restart file.
\begin{quote}\begin{description}
\item[{Use }] \leavevmode
\code{types\_mod} (\code{i8()}, \code{max\_num\_doms()}, \code{max\_files()}, \code{r8()}, \code{obstypelength()}), \code{state\_structure\_mod} (\code{get\_num\_domains()}), \code{ensemble\_manager\_mod} (\code{init\_ensemble\_manager()}, \code{get\_my\_num\_vars()}, \code{end\_ensemble\_manager()}, \code{ensemble\_type()}, \code{get\_my\_vars()}, \code{compute\_copy\_mean()}), \code{obs\_kind\_mod} (\code{get\_num\_quantities()}, \code{get\_name\_for\_quantity()}, \code{get\_index\_for\_quantity()}), \code{utilities\_mod} (\code{set\_multiple\_filename\_lists()}, \code{nmlfileunit()}, \code{close\_file()}, \code{open\_file()}, \code{check\_namelist\_read()}, \code{e\_msg()}, \code{e\_err()}, \code{error\_handler()}, \code{do\_nml\_file()}, \code{register\_module()}, \code{find\_namelist\_in\_file()}, \code{do\_nml\_term()}), \code{state\_vector\_io\_mod} (\code{read\_state()}), \code{time\_manager\_mod} (\code{set\_time\_missing()}, \code{print\_time()}, \code{time\_type()}), \code{sort\_mod} (\code{index\_sort()}), \code{mpi\_utilities\_mod} (\code{my\_task\_id()}, \code{send\_sum\_to()}, \code{finalize\_mpi\_utilities()}, \code{task\_count()}, \code{initialize\_mpi\_utilities()}), \code{location\_mod} (\code{location\_type()}), \code{assim\_model\_mod} (\code{get\_model\_size()}, \code{static\_init\_assim\_model()}, \code{get\_state\_meta\_data()}), \code{io\_filenames\_mod} (\code{read\_copy()}, \code{io\_filenames\_init()}, \code{set\_io\_copy\_flag()}, \code{get\_stage\_metadata()}, \code{file\_info\_type()}, \code{set\_member\_file\_metadata()}, \code{set\_file\_metadata()}, \code{get\_restart\_filename()}, \code{file\_info\_dump()}, \code{stage\_metadata\_type()})

\end{description}\end{quote}

\end{fulllineitems}



\chapter{Conversion process}
\label{docs/Process::doc}\label{docs/Process:conversion-process}
-- Working from Revision 12951 (12945 on rma\_trunk and trunk)


\section{1. Move from Subversion server to Github.}
\label{docs/Process:move-from-subversion-server-to-github}

\subsection{A. In docker container use git svn clone to create a local Git clone of the desired   Subversion branch/trunk.}
\label{docs/Process:a-in-docker-container-use-git-svn-clone-to-create-a-local-git-clone-of-the-desired-subversion-branch-trunk}\begin{quote}
\begin{quote}

Example:

\$ docker run -it niemacka/shpinx-git

\$ git svn clone https://svn-dares-dart.cgd.ucar.edu/DART/trunk/ --no-metadata  --no-minimize-url
\begin{itemize}
\item {} 
Use --revision \#\#\#\#\# to pull from a given revision number forward

\item {} 
For classic/trunk use revision 9994

\end{itemize}
\end{quote}
\end{quote}


\subsection{B. Create an empty GitHub repository under the desired account at github.com.}
\label{docs/Process:b-create-an-empty-github-repository-under-the-desired-account-at-github-com}

\subsection{C. Push local git clone to GitHub repo.}
\label{docs/Process:c-push-local-git-clone-to-github-repo}\begin{quote}
\begin{quote}

Example:

\$ git push --all https://github.com/account-name/repo-name
\begin{itemize}
\item {} 
Correct GitHub repo url will be at the top of empty repository.

\item {} 
This will always push to the Master branch of the repo.

\item {} 
Will require GitHub username and password

\end{itemize}
\end{quote}
\end{quote}


\subsection{D. It is possible that there will be an error that causes the push to fail because GitHub    does not allow files bigger than 100 MB. If that is the case these files will have to be removed in order to push to GitHub (They can be added later to the latest commit/revision but their commit/revision history cannot be kept). Use the git filter-branch command to remove necessary files.}
\label{docs/Process:d-it-is-possible-that-there-will-be-an-error-that-causes-the-push-to-fail-because-github-does-not-allow-files-bigger-than-100-mb-if-that-is-the-case-these-files-will-have-to-be-removed-in-order-to-push-to-github-they-can-be-added-later-to-the-latest-commit-revision-but-their-commit-revision-history-cannot-be-kept-use-the-git-filter-branch-command-to-remove-necessary-files}\begin{quote}
\begin{quote}

Example:

\$ git filter-branch --force --index-filter `git rm --cached --ignore-unmatch filename` --prune-empty --tag-name-filter cat -- --all
\begin{itemize}
\item {} 
For each file over 100 MB then...

\end{itemize}

\$ git for-each-ref --format=`delete \%(refname)` refs/original \textbar{} git update-ref --stdin

\$ git reflog expire --expire=now --all

\$ git gc --prune=now
\end{quote}
\end{quote}


\subsection{E. Now push to GitHub again and there should be no issues. (Use the —force option with git push).}
\label{docs/Process:e-now-push-to-github-again-and-there-should-be-no-issues-use-the-force-option-with-git-push}\begin{quote}
\begin{quote}

Example:

\$ git push --force --all  https://github.com/account-name/repo-name
\end{quote}
\end{quote}


\subsection{F. If the Subversion branch was not the current working branch create a new branch in Github cloned from the Master branch.}
\label{docs/Process:f-if-the-subversion-branch-was-not-the-current-working-branch-create-a-new-branch-in-github-cloned-from-the-master-branch}\begin{quote}
\begin{quote}

Example:

-- Move `trunk` to GitHub using the above steps. Then if trunk is not the current working branch click branch dropdown in top left of GitHub repository and type a new branch name (for example `classic`) that describes the branch. Then press enter. The new branch should now appear in the dropdown.
\begin{itemize}
\item {} 
At this point the master branch can be deleted if you so choose, or you can overwrite it by pushing another branch from subversion to GitHub using the steps above.

\item {} 
To overwrite the master branch use the --force option with git push

\end{itemize}
\end{quote}
\end{quote}


\section{2. Convert existing documentation to sphinx-doc style documentation. (This could be done either before or after the SVN branches have been pushed to GitHub.)}
\label{docs/Process:convert-existing-documentation-to-sphinx-doc-style-documentation-this-could-be-done-either-before-or-after-the-svn-branches-have-been-pushed-to-github}

\subsection{A. From the local machine create a clone of the new Github repository.}
\label{docs/Process:a-from-the-local-machine-create-a-clone-of-the-new-github-repository}\begin{quote}
\begin{quote}

Example:

\$ git clone https://github.com/account-name/repo-name
\end{quote}
\end{quote}


\subsection{B. Move all desired html files (or other convertible filetypes) that relate to documentation to a single directory.}
\label{docs/Process:b-move-all-desired-html-files-or-other-convertible-filetypes-that-relate-to-documentation-to-a-single-directory}

\subsection{C. To convert the html file to markdown (.md) which is used by sphinx use pandoc.}
\label{docs/Process:c-to-convert-the-html-file-to-markdown-md-which-is-used-by-sphinx-use-pandoc}\begin{quote}
\begin{quote}

Example:

\$ docker run -v \code{pwd}:/source jagregory/pandoc -f html -t markdown myfile.html -o myfile.md
\begin{itemize}
\item {} 
If there are a large number of files enter the file names you want to convert into a txt file with one name per line. Then use a python script to convert them.

\item {} 
Once the desired files are converted in pandoc you can delete the redundant html files.

\end{itemize}
\end{quote}
\end{quote}


\subsection{D. Next push the changes to GitHub.}
\label{docs/Process:d-next-push-the-changes-to-github}\begin{quote}
\begin{quote}

Example:

\$ git add —all

\$ git commit -m “converted html to md with pandoc”
\begin{itemize}
\item {} 
git commits require a message

\end{itemize}

\$ git push -u origin master
\begin{itemize}
\item {} 
Once the changes have been pushed to GitHub the local repository can be deleted if you so choose.

\end{itemize}
\end{quote}
\end{quote}


\subsection{E. Open an interactive docker container and clone the new Github repo.}
\label{docs/Process:e-open-an-interactive-docker-container-and-clone-the-new-github-repo}\begin{quote}
\begin{quote}

Example:

\$ docker run -it niemacka/sphinx

\$ git clone https://github.com/account-name/repo-name
\end{quote}
\end{quote}


\subsection{F. Choose a location for the sphinx-doc setup. Note that file paths are important for sphinx and particularly for the auto-documenting features. Sphinx likes to use paths relative to the location of the conf.py and index.rst files so it may be simplest to set these up in the root directory, although there is a way to setup an absolute path in the config (which I haven’t gotten to work with GitHub).}
\label{docs/Process:f-choose-a-location-for-the-sphinx-doc-setup-note-that-file-paths-are-important-for-sphinx-and-particularly-for-the-auto-documenting-features-sphinx-likes-to-use-paths-relative-to-the-location-of-the-conf-py-and-index-rst-files-so-it-may-be-simplest-to-set-these-up-in-the-root-directory-although-there-is-a-way-to-setup-an-absolute-path-in-the-config-which-i-havent-gotten-to-work-with-github}

\subsection{G. Once in the desired directory use the sphinx quickstart command to start an interactive process that will generate all the required files for sphinx-doc.}
\label{docs/Process:g-once-in-the-desired-directory-use-the-sphinx-quickstart-command-to-start-an-interactive-process-that-will-generate-all-the-required-files-for-sphinx-doc}\begin{quote}
\begin{quote}

Example:

\$ sphinx-quickstart

-- answer the questions presented and enable any feature that you intend to use
\begin{itemize}
\item {} 
Many elements can be left as default with no issues.

\item {} 
It is important to enable the python auto-doc extension

\item {} 
Enable github pages integration

\item {} 
It is important to create the Unix makefile

\end{itemize}
\end{quote}
\end{quote}


\subsection{H. There should now be a conf.py file that contains the configuration info that was set up during the quickstart. To allow sphinx to read markdown files add ‘recommonmark’ to the sphinx extensions section of the conf.py file. This should be near the top of the file.}
\label{docs/Process:h-there-should-now-be-a-conf-py-file-that-contains-the-configuration-info-that-was-set-up-during-the-quickstart-to-allow-sphinx-to-read-markdown-files-add-recommonmark-to-the-sphinx-extensions-section-of-the-conf-py-file-this-should-be-near-the-top-of-the-file}\begin{quote}
\begin{quote}

Eample:

extensions = {[}

`sphinx.ext.autodoc`,

`sphinx.ext.githubpages`,

‘recommonmark’,

{]}
\end{quote}
\end{quote}


\subsection{I. There is also an index.rst file (or whatever name it was given during the quick start). This file contains the Table of Contents Tree (toctree) which is where you can list the markdown files that you wish to include in your documentation using their relative path to the index.rst file and excluding the .md extension.}
\label{docs/Process:i-there-is-also-an-index-rst-file-or-whatever-name-it-was-given-during-the-quick-start-this-file-contains-the-table-of-contents-tree-toctree-which-is-where-you-can-list-the-markdown-files-that-you-wish-to-include-in-your-documentation-using-their-relative-path-to-the-index-rst-file-and-excluding-the-md-extension}\begin{quote}
\begin{quote}

Example:
\begin{itemize}
\item {} 
First install vim in the container

\end{itemize}

\$ apt-get update

\$ apt-get install vim -y

\$ vi index.rst

-- Should see something like this…

.. toctree::
:maxdepth: 2

-- Just list the files you want to include…

.. toctree::
.:maxdepth: 2

usage/installation

usage/quickstart

...
\end{quote}
\end{quote}


\subsection{J. Once the toctree has been updated. Use the commands from the makefile to create an html version and a pdf version of the documents.}
\label{docs/Process:j-once-the-toctree-has-been-updated-use-the-commands-from-the-makefile-to-create-an-html-version-and-a-pdf-version-of-the-documents}\begin{quote}
\begin{quote}

Example:

\$ make html

\$ make latexpdf
\begin{itemize}
\item {} 
If no other changes have been made to the documents both of these commands could generate a lot of warnings (which should be related to incorrect references to other internal documentation and images) They will build if you are running in an interactive container but these errors will stop circleci so it is important to check all references and images in a document and confirm there are no errors before integration with circleci.

\item {} 
You can verify that they built by checking the \_build directory for a latex and html subdirectory.

\item {} 
Some issue may be resolved by running make html or make latexpdf multiple times.

\end{itemize}
\end{quote}
\end{quote}


\subsection{K. Push the changes to GitHub like in step d.}
\label{docs/Process:k-push-the-changes-to-github-like-in-step-d}\begin{quote}
\begin{quote}
\begin{itemize}
\item {} 
When you try to commit from within a docker container you will get a message asking you to set an email and name to assign to the commit by adding to the git config.

\end{itemize}

Example:

\$ git config --global user.email ``you@example.com``

\$ git config --global user.name ``Your Name``
\end{quote}
\end{quote}


\section{3. Set up the sphinx-fortran extension. (This can all be done on GitHub or in a local clone of the repository but the process is the same.)}
\label{docs/Process:set-up-the-sphinx-fortran-extension-this-can-all-be-done-on-github-or-in-a-local-clone-of-the-repository-but-the-process-is-the-same}

\subsection{A. Open the conf.py configuration file and add the sphinx fortran extensions to the list of sphinx extensions this should be near the top of the document.}
\label{docs/Process:a-open-the-conf-py-configuration-file-and-add-the-sphinx-fortran-extensions-to-the-list-of-sphinx-extensions-this-should-be-near-the-top-of-the-document}\begin{quote}
\begin{quote}

Example:

-- It should look something like this…
extensions = {[}

`sphinx.ext.autodoc`,

`sphinx.ext.todo`,

`sphinx.ext.githubpages`,
{]}

-- Then add sphinxfortran.fortran\_domain and sphinxfortran.fortran\_autodoc…
extensions = {[}

`sphinx.ext.autodoc`,

`sphinx.ext.todo`,

`sphinx.ext.githubpages`,

`sphinxfortran.fortran\_domain`,

`sphinxfortran.fortran\_autodoc`,

{]}
\end{quote}
\end{quote}


\subsection{B. Below the extensions add a source for the fortran files to be parsed. This can be a list of files or a relative path using wildcards.}
\label{docs/Process:b-below-the-extensions-add-a-source-for-the-fortran-files-to-be-parsed-this-can-be-a-list-of-files-or-a-relative-path-using-wildcards}\begin{quote}
\begin{quote}

Example:

fortran\_src = {[}‘relativepath1/specific-file.f90’, ‘relativepath2/specific-file2.f90’{]}

or

fortran\_src = {[}‘relativepath/*.f90’{]}
\end{quote}
\end{quote}


\subsection{C. All setup should now be complete and the sphinx-fortran extension commands can now be used to document fortran programs, modules, etc…}
\label{docs/Process:c-all-setup-should-now-be-complete-and-the-sphinx-fortran-extension-commands-can-now-be-used-to-document-fortran-programs-modules-etc}\begin{quote}
\begin{quote}

Example:

-- In a md document that is included in the toctree add a line like the examples below to auto-document that element.

.. f:autoprogram:: program-name

.. f:autofunction:: {[}modname{]}/funcname

.. f:autosubroutine:: {[}modname/{]}/subrname

-- To document all programs, functions, and subroutines in a source file you can use…

..f:autosrcfile:: pathname
\begin{itemize}
\item {} 
Important to note auto-documenting will include descriptive comments about the element but only if they start on the line immediately below where the element is being declared.

\end{itemize}
\end{quote}
\end{quote}
\begin{itemize}
\item {} 
program example

\item {} 
!\textgreater{} example description must start here or it will not be included in the

\item {} 
!\textgreater{} documentation.

\end{itemize}
\begin{quote}
\begin{quote}

program example

!\textgreater{} This doesn’t work
\end{quote}
\end{quote}


\subsection{D. Now when the html or pdf sphinx files are generated the auto-documentation should be included.}
\label{docs/Process:d-now-when-the-html-or-pdf-sphinx-files-are-generated-the-auto-documentation-should-be-included}\begin{quote}
\begin{itemize}
\item {} 
You may need to use the make html and make latexpdf commands multiple times.

\end{itemize}
\end{quote}


\chapter{Indices and tables}
\label{index:indices-and-tables}\begin{itemize}
\item {} 
\DUrole{xref,std,std-ref}{genindex}

\item {} 
\DUrole{xref,std,std-ref}{modindex}

\item {} 
\DUrole{xref,std,std-ref}{search}

\end{itemize}



\renewcommand{\indexname}{Index}
\printindex
\end{document}
